% Options for packages loaded elsewhere
\PassOptionsToPackage{unicode}{hyperref}
\PassOptionsToPackage{hyphens}{url}
%
\documentclass[
  ignorenonframetext,
]{beamer}
\usepackage{pgfpages}
\setbeamertemplate{caption}[numbered]
\setbeamertemplate{caption label separator}{: }
\setbeamercolor{caption name}{fg=normal text.fg}
\beamertemplatenavigationsymbolsempty
% Prevent slide breaks in the middle of a paragraph
\widowpenalties 1 10000
\raggedbottom
\setbeamertemplate{part page}{
  \centering
  \begin{beamercolorbox}[sep=16pt,center]{part title}
    \usebeamerfont{part title}\insertpart\par
  \end{beamercolorbox}
}
\setbeamertemplate{section page}{
  \centering
  \begin{beamercolorbox}[sep=12pt,center]{part title}
    \usebeamerfont{section title}\insertsection\par
  \end{beamercolorbox}
}
\setbeamertemplate{subsection page}{
  \centering
  \begin{beamercolorbox}[sep=8pt,center]{part title}
    \usebeamerfont{subsection title}\insertsubsection\par
  \end{beamercolorbox}
}
\AtBeginPart{
  \frame{\partpage}
}
\AtBeginSection{
  \ifbibliography
  \else
    \frame{\sectionpage}
  \fi
}
\AtBeginSubsection{
  \frame{\subsectionpage}
}
\usepackage{lmodern}
\usepackage{amssymb,amsmath}
\usepackage{ifxetex,ifluatex}
\ifnum 0\ifxetex 1\fi\ifluatex 1\fi=0 % if pdftex
  \usepackage[T1]{fontenc}
  \usepackage[utf8]{inputenc}
  \usepackage{textcomp} % provide euro and other symbols
\else % if luatex or xetex
  \usepackage{unicode-math}
  \defaultfontfeatures{Scale=MatchLowercase}
  \defaultfontfeatures[\rmfamily]{Ligatures=TeX,Scale=1}
\fi
% Use upquote if available, for straight quotes in verbatim environments
\IfFileExists{upquote.sty}{\usepackage{upquote}}{}
\IfFileExists{microtype.sty}{% use microtype if available
  \usepackage[]{microtype}
  \UseMicrotypeSet[protrusion]{basicmath} % disable protrusion for tt fonts
}{}
\makeatletter
\@ifundefined{KOMAClassName}{% if non-KOMA class
  \IfFileExists{parskip.sty}{%
    \usepackage{parskip}
  }{% else
    \setlength{\parindent}{0pt}
    \setlength{\parskip}{6pt plus 2pt minus 1pt}}
}{% if KOMA class
  \KOMAoptions{parskip=half}}
\makeatother
\usepackage{xcolor}
\IfFileExists{xurl.sty}{\usepackage{xurl}}{} % add URL line breaks if available
\IfFileExists{bookmark.sty}{\usepackage{bookmark}}{\usepackage{hyperref}}
\hypersetup{
  pdftitle={Tema 0 - Preliminares},
  pdfauthor={Juan Gabriel Gomila \& María Santos},
  hidelinks,
  pdfcreator={LaTeX via pandoc}}
\urlstyle{same} % disable monospaced font for URLs
\newif\ifbibliography
\usepackage{color}
\usepackage{fancyvrb}
\newcommand{\VerbBar}{|}
\newcommand{\VERB}{\Verb[commandchars=\\\{\}]}
\DefineVerbatimEnvironment{Highlighting}{Verbatim}{commandchars=\\\{\}}
% Add ',fontsize=\small' for more characters per line
\usepackage{framed}
\definecolor{shadecolor}{RGB}{248,248,248}
\newenvironment{Shaded}{\begin{snugshade}}{\end{snugshade}}
\newcommand{\AlertTok}[1]{\textcolor[rgb]{0.94,0.16,0.16}{#1}}
\newcommand{\AnnotationTok}[1]{\textcolor[rgb]{0.56,0.35,0.01}{\textbf{\textit{#1}}}}
\newcommand{\AttributeTok}[1]{\textcolor[rgb]{0.77,0.63,0.00}{#1}}
\newcommand{\BaseNTok}[1]{\textcolor[rgb]{0.00,0.00,0.81}{#1}}
\newcommand{\BuiltInTok}[1]{#1}
\newcommand{\CharTok}[1]{\textcolor[rgb]{0.31,0.60,0.02}{#1}}
\newcommand{\CommentTok}[1]{\textcolor[rgb]{0.56,0.35,0.01}{\textit{#1}}}
\newcommand{\CommentVarTok}[1]{\textcolor[rgb]{0.56,0.35,0.01}{\textbf{\textit{#1}}}}
\newcommand{\ConstantTok}[1]{\textcolor[rgb]{0.00,0.00,0.00}{#1}}
\newcommand{\ControlFlowTok}[1]{\textcolor[rgb]{0.13,0.29,0.53}{\textbf{#1}}}
\newcommand{\DataTypeTok}[1]{\textcolor[rgb]{0.13,0.29,0.53}{#1}}
\newcommand{\DecValTok}[1]{\textcolor[rgb]{0.00,0.00,0.81}{#1}}
\newcommand{\DocumentationTok}[1]{\textcolor[rgb]{0.56,0.35,0.01}{\textbf{\textit{#1}}}}
\newcommand{\ErrorTok}[1]{\textcolor[rgb]{0.64,0.00,0.00}{\textbf{#1}}}
\newcommand{\ExtensionTok}[1]{#1}
\newcommand{\FloatTok}[1]{\textcolor[rgb]{0.00,0.00,0.81}{#1}}
\newcommand{\FunctionTok}[1]{\textcolor[rgb]{0.00,0.00,0.00}{#1}}
\newcommand{\ImportTok}[1]{#1}
\newcommand{\InformationTok}[1]{\textcolor[rgb]{0.56,0.35,0.01}{\textbf{\textit{#1}}}}
\newcommand{\KeywordTok}[1]{\textcolor[rgb]{0.13,0.29,0.53}{\textbf{#1}}}
\newcommand{\NormalTok}[1]{#1}
\newcommand{\OperatorTok}[1]{\textcolor[rgb]{0.81,0.36,0.00}{\textbf{#1}}}
\newcommand{\OtherTok}[1]{\textcolor[rgb]{0.56,0.35,0.01}{#1}}
\newcommand{\PreprocessorTok}[1]{\textcolor[rgb]{0.56,0.35,0.01}{\textit{#1}}}
\newcommand{\RegionMarkerTok}[1]{#1}
\newcommand{\SpecialCharTok}[1]{\textcolor[rgb]{0.00,0.00,0.00}{#1}}
\newcommand{\SpecialStringTok}[1]{\textcolor[rgb]{0.31,0.60,0.02}{#1}}
\newcommand{\StringTok}[1]{\textcolor[rgb]{0.31,0.60,0.02}{#1}}
\newcommand{\VariableTok}[1]{\textcolor[rgb]{0.00,0.00,0.00}{#1}}
\newcommand{\VerbatimStringTok}[1]{\textcolor[rgb]{0.31,0.60,0.02}{#1}}
\newcommand{\WarningTok}[1]{\textcolor[rgb]{0.56,0.35,0.01}{\textbf{\textit{#1}}}}
\usepackage{graphicx,grffile}
\makeatletter
\def\maxwidth{\ifdim\Gin@nat@width>\linewidth\linewidth\else\Gin@nat@width\fi}
\def\maxheight{\ifdim\Gin@nat@height>\textheight\textheight\else\Gin@nat@height\fi}
\makeatother
% Scale images if necessary, so that they will not overflow the page
% margins by default, and it is still possible to overwrite the defaults
% using explicit options in \includegraphics[width, height, ...]{}
\setkeys{Gin}{width=\maxwidth,height=\maxheight,keepaspectratio}
% Set default figure placement to htbp
\makeatletter
\def\fps@figure{htbp}
\makeatother
\setlength{\emergencystretch}{3em} % prevent overfull lines
\providecommand{\tightlist}{%
  \setlength{\itemsep}{0pt}\setlength{\parskip}{0pt}}
\setcounter{secnumdepth}{-\maxdimen} % remove section numbering

\title{Tema 0 - Preliminares}
\author{Juan Gabriel Gomila \& María Santos}
\date{null}

\begin{document}
\frame{\titlepage}

\hypertarget{cuerpos}{%
\section{Cuerpos}\label{cuerpos}}

\begin{frame}{La estructura de cuerpo}
\protect\hypertarget{la-estructura-de-cuerpo}{}

Cuerpo. Sea \(\mathbb{K}\) un conjunto dotado de dos operaciones,
adición (\(+\)) y multiplicación (\(\cdot\)). Diremos que \(\mathbb{K}\)
es un cuerpo si para todo \(a,b\in\mathbb{K}\) se cumplen las
condiciones siguientes:

\begin{itemize}
\tightlist
\item
  \(+\) y \(\cdot\) son operaciones internas sobre \(\mathbb{K}\):
  \(a+b\in\mathbb{K}\) y \(a\cdot b\in\mathbb{K}\)
\item
  \(+\) y \(\cdot\) son operaciones conmutativas: \(a+b=b+a\) y
  \(a\cdot b=b\cdot a\)
\item
  \(+\) y \(\cdot\) son operaciones asociativas: \((a+b)+c=a+(b+c)\) y
  \((a\cdot b)\cdot c=a\cdot(b\cdot c)\)
\item
  Hay un elemento neutro para la adición:
  \(a+0=0+a=a\quad \forall a\in\mathbb{K}\)
\item
  Hay un elemento neutro para la multiplicación (distinto del neutro de
  la adición): \(a\cdot 1=1\cdot a=a\quad \forall a\in\mathbb{K}\)
\end{itemize}

\end{frame}

\begin{frame}{La estructura de cuerpo}
\protect\hypertarget{la-estructura-de-cuerpo-1}{}

\begin{itemize}
\tightlist
\item
  Elemento opuesto: \(\forall a\in\mathbb{K}\) hay otro elemento
  \(-a\in\mathbb{K}\) tal que \(a+(-a)=(-a)+a=0\)
\item
  Elemento inverso: \(\forall a\in\mathbb{K},\ a\ne 0\) hay otro
  elemento \(a^{-1}\in\mathbb{K}\) tal que
  \(a\cdot a^{-1}=a^{-1}\cdot a=1\)
\item
  La operación \(\cdot\) es distributiva respecto \(+\):
  \(a\cdot(b+c)=a\cdot b+a\cdot c\)
\end{itemize}

Observación. Cuando no pueda haber confusión quitaremos el signo
\(\cdot\) para denotar la operación de multiplicación. Es decir,
denotaremos \(a\cdot b\) como \(ab\).

\end{frame}

\begin{frame}{Propiedades de los cuerpos}
\protect\hypertarget{propiedades-de-los-cuerpos}{}

Propiedades de los cuerpos. En un cuerpo \(\mathbb{K}\) se verifican las
siguientes propiedades:

\begin{itemize}
\tightlist
\item
  Propiedad de simplificación para la suma: \(a+b=a+c\Rightarrow\ b=c\)
\item
  Los neutros (0 y 1) son únicos
\item
  Cada elemento tienen un único opuesto
\item
  Cada elemento diferente de 0 tiene un único inverso
\item
  0 es absorbente para la multiplicación:
  \(a\cdot 0 = 0\quad \forall a\in\mathbb{K}\)
\item
  \(\mathbb{K}\) no tiene divisores de 0: \(ab=0\Rightarrow a=0\) o
  \(b=0\)
\end{itemize}

¡Atención! La demostración de esta proposición se encuentra en pdf. Este
pdf lo podréis encontrar en el Github, en la carpeta demostraciones, o
bien como material de esta clase.

\end{frame}

\begin{frame}{Cuerpos conocidos}
\protect\hypertarget{cuerpos-conocidos}{}

\textbf{Ejemplo 1}

Algunos de los cuerpos más conocidos son:

\begin{itemize}
\tightlist
\item
  \(\mathbb{Z}_2=\{0,1\}\): Cuerpo finito de dos elementos
\item
  \(\mathbb{Q}\): Los números racionales

  \begin{itemize}
  \tightlist
  \item
    suma: \(\frac{a}{b}+\frac{c}{d} = \frac{ad+bc}{bd}\)
    \(\qquad a,b,c,d\in\mathbb{Z}\)
  \item
    producto: \(\frac{a}{b}\cdot\frac{c}{d} = \frac{ac}{bd}\)
    \(\qquad a,b,c,d\in\mathbb{Z}\)
  \end{itemize}
\item
  \(\mathbb{R}\): Los números reales
\item
  \(\mathbb{C}\): Los números complejos

  \begin{itemize}
  \tightlist
  \item
    suma: \((a+bi)+(c+di) = (a+c)+(b+d)i\)
    \(\qquad a,b,c,d\in\mathbb{R}\)
  \item
    producto: \((a+bi)\cdot(c+di) = (ac-bd)+(ad+bc)i\)
    \(\qquad a,b,c,d\in\mathbb{R}\)
  \end{itemize}
\end{itemize}

\textbf{Ejemplo 2}

Los números naturales, \(\mathbb{N}:=\{0,1,2,...\}\), no son un cuerpo.
No hay elementos opuestos para ningún elemento del conjunto.

\end{frame}

\begin{frame}{El cuerpo \(\mathbb{Z}_2\)}
\protect\hypertarget{el-cuerpo-mathbbz_2}{}

Entremos un poquito más en detalle en este cuerpo tan interesante:

\begin{itemize}
\tightlist
\item
  Consta de 2 elementos: el 0 y el 1
\item
  Sus tablas de suma y producto son las siguientes
\end{itemize}

\begin{figure}
\centering
\includegraphics{Images/Sum.gif}
\caption{Tabla de la suma}
\end{figure}

\begin{figure}
\centering
\includegraphics{Images/Prod.gif}
\caption{Tabla del producto}
\end{figure}

\end{frame}

\hypertarget{nuxfameros-complejos}{%
\section{Números complejos}\label{nuxfameros-complejos}}

\begin{frame}{Números complejos}
\protect\hypertarget{nuxfameros-complejos-1}{}

Conjunto de Números Complejos. \(\mathbb{C}=\{(a,b):a,b\in\mathbb{R}\}\)
dotado de las operaciones:

\begin{itemize}
\tightlist
\item
  suma: \((a,b)+(c,d) = (a+c,b+d)\) \(\qquad a,b,c,d\in\mathbb{R}\)
\item
  producto: \((a,b)\cdot(c,d) = (ac-bd,ad+bc)\)
  \(\qquad a,b,c,d\in\mathbb{R}\)
\end{itemize}

\end{frame}

\begin{frame}{Números complejos}
\protect\hypertarget{nuxfameros-complejos-2}{}

Forma Binómica. Si \(z\in\mathbb{C}\) tal que \(z=(a,b)\), su forma
binómica es \(z=a+bi\)

La forma binómica aparece al definir la unidad imaginaria, \(i\):

Unidad Imaginaria. \(i=(0,1)\)

Entonces, si \(z=(a,b)\), tenemos que
\[z=(a,b)=(a,0)+(0,b)=(a,0)+(b,0)\cdot(0,1)=a+bi\]

\end{frame}

\begin{frame}{Números complejos}
\protect\hypertarget{nuxfameros-complejos-3}{}

Parte Real. Si \(z=a+bi\), Re\((z)=a\)

Parte Imaginaria. Si \(z=a+bi\), Im\((z)=b\)

Conjugado de z. Si \(z=a+bi\), \(\bar{z}=a-bi\)

Módulo. \(|z|=\sqrt{z\cdot\bar{z}}\)

Argumento. Si \(z=a+bi\),
\(\text{arg}(z)=\arctan\left(\frac{b}{a}\right)\). Se da en radianes.

Argumento principal. \(\text{Arg}(z)\in(-\pi,\pi]\)

\end{frame}

\begin{frame}{Plano Complejo}
\protect\hypertarget{plano-complejo}{}

Los números complejos se suelen representar en un plano, denominado
plano complejo, donde el eje de las abcisas es el eje Real y, el de las
ordenadas, el eje Imaginario

\includegraphics{Images/complex.png}

\end{frame}

\begin{frame}{Forma polar}
\protect\hypertarget{forma-polar}{}

Fórmula de Euler. \(e^{i\theta}=\cos(\theta)+i\sin(\theta)\)

Forma polar. \(z=re^{i\varphi}\) donde \(r = |z|\) y
\(\varphi = \text{arg}(z)\)

\includegraphics{Images/comlpex2.png}

\end{frame}

\hypertarget{nuxfameros-complejos-con-r}{%
\section{\texorpdfstring{Números complejos con
\texttt{R}}{Números complejos con R}}\label{nuxfameros-complejos-con-r}}

\begin{frame}[fragile]{Números complejos con \texttt{R}}
\protect\hypertarget{nuxfameros-complejos-con-r-1}{}

Podemos definir números complejos de diferentes formas:

\begin{Shaded}
\begin{Highlighting}[]
\NormalTok{z1 =}\StringTok{ }\DecValTok{2}\OperatorTok{+}\NormalTok{1i }\CommentTok{#Definimos el complejo en forma binómica}
\NormalTok{z1}
\end{Highlighting}
\end{Shaded}

\begin{verbatim}
[1] 2+1i
\end{verbatim}

\begin{Shaded}
\begin{Highlighting}[]
\NormalTok{z2 =}\StringTok{ }\KeywordTok{complex}\NormalTok{(}\DataTypeTok{real =} \DecValTok{2}\NormalTok{, }\DataTypeTok{imaginary =} \DecValTok{-1}\NormalTok{) }\CommentTok{#Definimos mediante parte real e imaginaria}
\NormalTok{z2}
\end{Highlighting}
\end{Shaded}

\begin{verbatim}
[1] 2-1i
\end{verbatim}

\begin{Shaded}
\begin{Highlighting}[]
\NormalTok{z3 =}\StringTok{ }\KeywordTok{complex}\NormalTok{(}\DataTypeTok{modulus =} \DecValTok{2}\NormalTok{, }\DataTypeTok{argument =}\NormalTok{ pi) }\CommentTok{#Definimos mediante módulo y argumento}
\NormalTok{z3}
\end{Highlighting}
\end{Shaded}

\begin{verbatim}
[1] -2+0i
\end{verbatim}

\end{frame}

\begin{frame}[fragile]{Números complejos con \texttt{R}}
\protect\hypertarget{nuxfameros-complejos-con-r-2}{}

¡Observación! Si queremos escribir los números complejos \(a+i\) o
\(a-i\) en \texttt{R}, donde \(a\) puede ser cualquier número real, lo
tenemos que hacer del siguiente modo: \texttt{a+1i} o \texttt{a-1i}, ya
que si no la consola nos devolverá error.

\end{frame}

\begin{frame}[fragile]{Números complejos con \texttt{R}}
\protect\hypertarget{nuxfameros-complejos-con-r-3}{}

La función \texttt{typeof()} es útil a la hora de comprobar el tipo de
dato con el que estamos trabajando:

\begin{Shaded}
\begin{Highlighting}[]
\KeywordTok{typeof}\NormalTok{(z1)}
\end{Highlighting}
\end{Shaded}

\begin{verbatim}
[1] "complex"
\end{verbatim}

\begin{Shaded}
\begin{Highlighting}[]
\KeywordTok{typeof}\NormalTok{(z2)}
\end{Highlighting}
\end{Shaded}

\begin{verbatim}
[1] "complex"
\end{verbatim}

\begin{Shaded}
\begin{Highlighting}[]
\KeywordTok{typeof}\NormalTok{(z3)}
\end{Highlighting}
\end{Shaded}

\begin{verbatim}
[1] "complex"
\end{verbatim}

\end{frame}

\begin{frame}[fragile]{Números complejos con \texttt{R}}
\protect\hypertarget{nuxfameros-complejos-con-r-4}{}

Para obtener la parte real y la parte imaginaria de cualquier número
complejo, utilizamos, respectivamente, las funciones \texttt{Re()} e
\texttt{Im()}:

\begin{Shaded}
\begin{Highlighting}[]
\CommentTok{#Parte Real}
\KeywordTok{Re}\NormalTok{(z1)}
\end{Highlighting}
\end{Shaded}

\begin{verbatim}
[1] 2
\end{verbatim}

\begin{Shaded}
\begin{Highlighting}[]
\CommentTok{#Parte Imaginaria}
\KeywordTok{Im}\NormalTok{(z3)}
\end{Highlighting}
\end{Shaded}

\begin{verbatim}
[1] 2.449294e-16
\end{verbatim}

\end{frame}

\begin{frame}[fragile]{Números complejos con \texttt{R}}
\protect\hypertarget{nuxfameros-complejos-con-r-5}{}

El conjugado de un número complejo se obtiene mediante la función
\texttt{Conj()}:

\begin{Shaded}
\begin{Highlighting}[]
\KeywordTok{Conj}\NormalTok{(z1)}
\end{Highlighting}
\end{Shaded}

\begin{verbatim}
[1] 2-1i
\end{verbatim}

\begin{Shaded}
\begin{Highlighting}[]
\KeywordTok{Conj}\NormalTok{(z2)}
\end{Highlighting}
\end{Shaded}

\begin{verbatim}
[1] 2+1i
\end{verbatim}

\begin{Shaded}
\begin{Highlighting}[]
\KeywordTok{Conj}\NormalTok{(z3)}
\end{Highlighting}
\end{Shaded}

\begin{verbatim}
[1] -2-0i
\end{verbatim}

\end{frame}

\begin{frame}[fragile]{Números complejos con \texttt{R}}
\protect\hypertarget{nuxfameros-complejos-con-r-6}{}

Para obtener el módulo y el argumento (principal) de cualquier número
complejo, utilizamos, respectivamente, las funciones \texttt{Mod()} y
\texttt{Arg()}:

\begin{Shaded}
\begin{Highlighting}[]
\CommentTok{#Módulo}
\KeywordTok{Mod}\NormalTok{(z2)}
\end{Highlighting}
\end{Shaded}

\begin{verbatim}
[1] 2.236068
\end{verbatim}

\begin{Shaded}
\begin{Highlighting}[]
\CommentTok{#Argumento principal}
\KeywordTok{Arg}\NormalTok{(z3)}
\end{Highlighting}
\end{Shaded}

\begin{verbatim}
[1] 3.141593
\end{verbatim}

\end{frame}

\begin{frame}[fragile]{Números complejos con \texttt{R}}
\protect\hypertarget{nuxfameros-complejos-con-r-7}{}

Las operaciones básicas con números complejos se realizan del siguiente
modo:

\begin{Shaded}
\begin{Highlighting}[]
\NormalTok{z1}\OperatorTok{+}\NormalTok{z2 }\CommentTok{#Suma de números complejos}
\end{Highlighting}
\end{Shaded}

\begin{verbatim}
[1] 4+0i
\end{verbatim}

\begin{Shaded}
\begin{Highlighting}[]
\DecValTok{3}\OperatorTok{*}\NormalTok{z3 }\CommentTok{#Producto por un escalar}
\end{Highlighting}
\end{Shaded}

\begin{verbatim}
[1] -6+0i
\end{verbatim}

\begin{Shaded}
\begin{Highlighting}[]
\NormalTok{z2}\OperatorTok{*}\NormalTok{z3 }\CommentTok{#Producto de números complejos}
\end{Highlighting}
\end{Shaded}

\begin{verbatim}
[1] -4+2i
\end{verbatim}

\end{frame}

\hypertarget{trabajando-con-python-en-markdown}{%
\section{\texorpdfstring{Trabajando con \texttt{Python} en
Markdown}{Trabajando con Python en Markdown}}\label{trabajando-con-python-en-markdown}}

\begin{frame}[fragile]{Trabajando con \texttt{Python} en Markdown}
\protect\hypertarget{trabajando-con-python-en-markdown-1}{}

En primer lugar, tendréis que instalar el paquete de \texttt{R} llamado
\texttt{reticulate} del siguiente modo:

\texttt{install.packages("reticulate")}

Si en algúm momento necesitáis instalar una librería de \texttt{Python}
en \texttt{Rstudio}, se debe ejecutar la siguiente función:

\texttt{py\_install("NombreDelPaquete")}

\end{frame}

\begin{frame}[fragile]{Trabajando con \texttt{Python} en Markdown}
\protect\hypertarget{trabajando-con-python-en-markdown-2}{}

Para poder utilizar \texttt{Python} en un Markdown, en el chunk de
ajustes deberéis añadir las instrucciones que se muestran a continuación

\begin{itemize}
\tightlist
\item
  \texttt{library(reticulate)}
\item
  \texttt{use\_python("/anaconda3/bin/python3")}
\end{itemize}

La primera para cargar la librería \texttt{reticulate} y la segunda para
ubicar donde está \texttt{Python} en nuestro ordenador

\end{frame}

\hypertarget{nuxfameros-complejos-con-python}{%
\section{\texorpdfstring{Números complejos con
\texttt{Python}}{Números complejos con Python}}\label{nuxfameros-complejos-con-python}}

\begin{frame}[fragile]{Números complejos con \texttt{Python}}
\protect\hypertarget{nuxfameros-complejos-con-python-1}{}

Podemos definir números complejos de diferentes formas:

\begin{Shaded}
\begin{Highlighting}[]
\NormalTok{z1 }\OperatorTok{=} \DecValTok{4}\OperatorTok{+}\NormalTok{3j}
\NormalTok{z1}
\end{Highlighting}
\end{Shaded}

\begin{verbatim}
(4+3j)
\end{verbatim}

\begin{Shaded}
\begin{Highlighting}[]
\NormalTok{z2 }\OperatorTok{=} \BuiltInTok{complex}\NormalTok{(}\DecValTok{1}\NormalTok{,}\DecValTok{7}\NormalTok{)}
\NormalTok{z2}
\end{Highlighting}
\end{Shaded}

\begin{verbatim}
(1+7j)
\end{verbatim}

\end{frame}

\begin{frame}[fragile]{Números complejos con \texttt{Python}}
\protect\hypertarget{nuxfameros-complejos-con-python-2}{}

¡Observación! Si queremos escribir los números complejos \(a+i\) o
\(a-i\) en \texttt{Python}, donde \(a\) puede ser cualquier número real,
lo tenemos que hacer del siguiente modo: \texttt{a+ji} o \texttt{a-ji},
ya que si no la consola nos devolverá error.

\end{frame}

\begin{frame}[fragile]{Números complejos con \texttt{Python}}
\protect\hypertarget{nuxfameros-complejos-con-python-3}{}

La función \texttt{type()} es útil a la hora de comprobar el tipo de
dato con el que estamos trabajando:

\begin{Shaded}
\begin{Highlighting}[]
\BuiltInTok{type}\NormalTok{(z1)}
\end{Highlighting}
\end{Shaded}

\begin{verbatim}
<class 'complex'>
\end{verbatim}

\begin{Shaded}
\begin{Highlighting}[]
\BuiltInTok{type}\NormalTok{(z2)}
\end{Highlighting}
\end{Shaded}

\begin{verbatim}
<class 'complex'>
\end{verbatim}

\end{frame}

\begin{frame}[fragile]{Números complejos con \texttt{Python}}
\protect\hypertarget{nuxfameros-complejos-con-python-4}{}

Para obtener la parte real y la parte imaginaria de cualquier número
complejo, utilizamos, \texttt{.real} y \texttt{.imag}

\begin{Shaded}
\begin{Highlighting}[]
\NormalTok{z1.real }\CommentTok{#Parte real}
\end{Highlighting}
\end{Shaded}

\begin{verbatim}
4.0
\end{verbatim}

\begin{Shaded}
\begin{Highlighting}[]
\NormalTok{z2.imag }\CommentTok{#Parte imaginaria}
\end{Highlighting}
\end{Shaded}

\begin{verbatim}
7.0
\end{verbatim}

\end{frame}

\begin{frame}[fragile]{Números complejos con \texttt{Python}}
\protect\hypertarget{nuxfameros-complejos-con-python-5}{}

El conjugado de un número complejo se obtiene mediante
\texttt{.conjugate()}:

\begin{Shaded}
\begin{Highlighting}[]
\NormalTok{z1.conjugate()}
\end{Highlighting}
\end{Shaded}

\begin{verbatim}
(4-3j)
\end{verbatim}

\begin{Shaded}
\begin{Highlighting}[]
\NormalTok{z2.conjugate()}
\end{Highlighting}
\end{Shaded}

\begin{verbatim}
(1-7j)
\end{verbatim}

\end{frame}

\begin{frame}[fragile]{Números complejos con \texttt{Python}}
\protect\hypertarget{nuxfameros-complejos-con-python-6}{}

Para obtener el módulo y el argumento (principal) de cualquier número
complejo, utilizamos, respectivamente, las funciones \texttt{abs()} y
\texttt{cmath.phase()}:

\begin{Shaded}
\begin{Highlighting}[]
\ImportTok{import}\NormalTok{ cmath}
\BuiltInTok{abs}\NormalTok{(z1)}
\end{Highlighting}
\end{Shaded}

\begin{verbatim}
5.0
\end{verbatim}

\begin{Shaded}
\begin{Highlighting}[]
\NormalTok{cmath.phase(z2)}
\end{Highlighting}
\end{Shaded}

\begin{verbatim}
1.4288992721907328
\end{verbatim}

\end{frame}

\begin{frame}[fragile]{Números complejos con \texttt{Python}}
\protect\hypertarget{nuxfameros-complejos-con-python-7}{}

Las operaciones básicas con números complejos se realizan del siguiente
modo:

\begin{Shaded}
\begin{Highlighting}[]
\NormalTok{z1}\OperatorTok{+}\NormalTok{z2 }\CommentTok{#Suma de numeros complejos}
\end{Highlighting}
\end{Shaded}

\begin{verbatim}
(5+10j)
\end{verbatim}

\begin{Shaded}
\begin{Highlighting}[]
\DecValTok{5}\OperatorTok{*}\NormalTok{z2 }\CommentTok{#Producto por un escalar}
\end{Highlighting}
\end{Shaded}

\begin{verbatim}
(5+35j)
\end{verbatim}

\begin{Shaded}
\begin{Highlighting}[]
\NormalTok{z1}\OperatorTok{*}\NormalTok{z2 }\CommentTok{#Producto de numeros complejos}
\end{Highlighting}
\end{Shaded}

\begin{verbatim}
(-17+31j)
\end{verbatim}

\end{frame}

\hypertarget{trabajando-con-octave-en-markdown}{%
\section{\texorpdfstring{Trabajando con \texttt{Octave} en
Markdown}{Trabajando con Octave en Markdown}}\label{trabajando-con-octave-en-markdown}}

\begin{frame}[fragile]{Trabajando con \texttt{Octave} en Markdown}
\protect\hypertarget{trabajando-con-octave-en-markdown-1}{}

Para poder utilizar \texttt{Octave} en Markdown deberéis introducir en
el chunk de ajustes el siguiente código:

\texttt{knitr::opts\_chunk\$set(echo\ =\ TRUE,\ engine.path\ =\ list(\ octave\ =\ \textquotesingle{}/Applications/Octave-4.4.1.app/Contents/Resources/usr/bin/octave\textquotesingle{}))}

Lo que está entre comillas es la dirección donde se encuentra el
lenguaje \texttt{Octave} en nuestro ordenador

¡Ojo! A la hora de utilizar chunks de \texttt{Octave}, tendréis que
introducir las variables cada vez. Es decir, no se guarda la información
de un chunk a otro

\end{frame}

\hypertarget{nuxfameros-complejos-con-octave}{%
\section{\texorpdfstring{Números complejos con
\texttt{Octave}}{Números complejos con Octave}}\label{nuxfameros-complejos-con-octave}}

\begin{frame}[fragile]{Números complejos con \texttt{Octave}}
\protect\hypertarget{nuxfameros-complejos-con-octave-1}{}

Podemos definir números complejos de diferentes formas:

\begin{Shaded}
\begin{Highlighting}[]
\NormalTok{z1 }\OperatorTok{=} \FunctionTok{complex}\NormalTok{(}\FloatTok{1}\OperatorTok{,}\FloatTok{2}\NormalTok{)}
\NormalTok{z2 }\OperatorTok{=} \FloatTok{2}\OperatorTok{-}\FunctionTok{i}
\NormalTok{tipoDato1 }\OperatorTok{=} \FunctionTok{class}\NormalTok{(z1)}
\NormalTok{tipoDato2 }\OperatorTok{=} \FunctionTok{class}\NormalTok{(z2)}
\end{Highlighting}
\end{Shaded}

\begin{verbatim}
z1 =  1 + 2i
z2 =  2 - 1i
tipoDato1 = double
tipoDato2 = double
\end{verbatim}

La función \texttt{class()} es útil a la hora de comprobar el tipo de
dato con el que estamos trabajando.

\end{frame}

\begin{frame}[fragile]{Números complejos con \texttt{Octave}}
\protect\hypertarget{nuxfameros-complejos-con-octave-2}{}

Para obtener la parte real y la parte imaginaria de cualquier número
complejo, utilizamos, respectivamente, las funciones \texttt{real()} e
\texttt{imag()}

\begin{Shaded}
\begin{Highlighting}[]
\NormalTok{z1 }\OperatorTok{=} \FunctionTok{complex}\NormalTok{(}\FloatTok{1}\OperatorTok{,}\FloatTok{2}\NormalTok{)}\OperatorTok{;}
\NormalTok{z2 }\OperatorTok{=} \FloatTok{2}\OperatorTok{-}\FunctionTok{i}\OperatorTok{;}

\FunctionTok{real}\NormalTok{(z1)}
\FunctionTok{imag}\NormalTok{(z2)}
\end{Highlighting}
\end{Shaded}

\begin{verbatim}
ans =  1
ans = -1
\end{verbatim}

Observación. Para trabajar con los números complejos anteriormente
definidos, hemos tenido que crearlos de nuevo porque \texttt{Octave}
trata cada chunk por separado.

\end{frame}

\begin{frame}[fragile]{Números complejos con \texttt{Octave}}
\protect\hypertarget{nuxfameros-complejos-con-octave-3}{}

El conjugado de un número complejo se obtiene mediante la función
\texttt{conj()}:

\begin{Shaded}
\begin{Highlighting}[]
\NormalTok{z1 }\OperatorTok{=} \FunctionTok{complex}\NormalTok{(}\FloatTok{1}\OperatorTok{,}\FloatTok{2}\NormalTok{)}\OperatorTok{;}
\NormalTok{z2 }\OperatorTok{=} \FloatTok{2}\OperatorTok{-}\FunctionTok{i}\OperatorTok{;}

\FunctionTok{conj}\NormalTok{(z1)}
\FunctionTok{abs}\NormalTok{(z2)}
\FunctionTok{arg}\NormalTok{(z1)}
\FunctionTok{angle}\NormalTok{(z2)}
\end{Highlighting}
\end{Shaded}

\begin{verbatim}
ans =  1 - 2i
ans =  2.2361
ans =  1.1071
ans = -0.46365
\end{verbatim}

Para obtener el módulo y el argumento (principal) de cualquier número
complejo, utilizamos, respectivamente, las funciones \texttt{abs()} y
\texttt{arg()} o \texttt{angle()}.

\end{frame}

\begin{frame}[fragile]{Números complejos con \texttt{Octave}}
\protect\hypertarget{nuxfameros-complejos-con-octave-4}{}

Las operaciones básicas con números complejos se realizan del siguiente
modo:

\begin{Shaded}
\begin{Highlighting}[]
\NormalTok{z1 }\OperatorTok{=} \FunctionTok{complex}\NormalTok{(}\FloatTok{1}\OperatorTok{,}\FloatTok{2}\NormalTok{)}\OperatorTok{;}
\NormalTok{z2 }\OperatorTok{=} \FloatTok{2}\OperatorTok{-}\FunctionTok{i}\OperatorTok{;}

\CommentTok{#Suma de numeros complejos}
\NormalTok{z1}\OperatorTok{+}\NormalTok{z2}
\CommentTok{# Producto por un escalar}
\FloatTok{8}\OperatorTok{*}\NormalTok{z2}
\CommentTok{# Producto de numeros complejos}
\NormalTok{z1}\OperatorTok{*}\NormalTok{z2}
\end{Highlighting}
\end{Shaded}

\begin{verbatim}
ans =  3 + 1i
ans =  16 -  8i
ans =  4 + 3i
\end{verbatim}

\end{frame}

\hypertarget{polinomios}{%
\section{Polinomios}\label{polinomios}}

\begin{frame}{Polinomios}
\protect\hypertarget{polinomios-1}{}

Sea \(\mathbb{K}\) un cuerpo cualquiera

Polinomio en una variable. Objeto de la forma
\(p(x)=a_0+a_1x+\cdots+a_nx^n\) donde
\(a_i\in\mathbb{K}\quad \forall i=0,\dots,n\)

Polinomio iguales. Dos polinomios son iguales si tienen el mismo grado y
los mismos coeficientes. Es decir, dados
\(p(x) = a_0+a_1x+\cdots+a_nx^n\) y \(q(x) = b_0+b_1x+\cdots+b_mx^m\),
\(p\) y \(q\) son iguales si, y solo si, \(n = m\) y
\(a_i=b_i\quad \forall i=0,1,\dots,n\)

\end{frame}

\begin{frame}{Polinomios}
\protect\hypertarget{polinomios-2}{}

Grado de un polinomio. Si \(a_n\ne 0\) y
\(a_j = 0\ \forall j=n+1,n+2,\dots\), se dice que el grado del polinomio
es \(n\).

\textbf{Ejemplo 3}

\begin{itemize}
\tightlist
\item
  \(p(x)=x^2+5x+1\) es un polinomio de segundo grado (grado 2)
\item
  \(q(x)=x^4-5\) es un polinomio de grado 4
\end{itemize}

Polinomio 0. \(p(x)=0\). Es decir, \(a_i=0\) para todo \(i=0,\dots,n\)

Polinomio constante. Polinomio de grado 0

\textbf{Ejemplo 4}

\(p(x)=5\) es un polinomio constante

\end{frame}

\begin{frame}{Polinomios}
\protect\hypertarget{polinomios-3}{}

Conjunto de polinomios en una variable. Indicaremos por
\(\mathbb{K}[x]\) el conjunto de polinomios en una determinada variable
\(x\) y con coeficientes en \(\mathbb{K}\). Sobre \(\mathbb{K}[x]\) se
pueden considerar la adición y la multiplicación definidas a partir de
las operaciones de \(\mathbb{K}\) de la manera siguiente

\begin{itemize}
\tightlist
\item
  \(p(x)+q(x)\) es el polinomio que tiene por coeficientes la suma (en
  \(\mathbb{K}\)) de los coeficientes de \(p(x)\) y \(q(x)\). Más
  claramente, si \(p(x)=a_0+a_1x+\cdots+a_nx^n\) y
  \(q(x)=b_0+b_1x+\cdots+b_mx^m\), entonces
  \(p(x)+q(x)=(a_0+b_0)+(a_1+b_1)x+\cdots\)
\item
  \(p(x)q(x)\) es el polinomio \(c_0+c_1(x)+\cdots+c_{nm}x^{n+m}\) donde
  \(c_j=a_0b_j+a_1b_{j-1}+\cdots+a_jb_0\) \(\quad j=0,1,\dots,n+m\)
\end{itemize}

\end{frame}

\begin{frame}{Polinomios}
\protect\hypertarget{polinomios-4}{}

\textbf{Ejemplo 5}

Sean \(p(x) = x+1\) y \(q(x) = x-1\). Entonces,

\begin{itemize}
\tightlist
\item
  Su suma es \(p(x)+q(x)=(x+1)+(x-1) = (1+1)x+(1-1) = 2x+0=2x\)
\item
  Su producto es \(p(x)\cdot q(x)=(x+1)\cdot(x-1)=x^2+x-x-1=x^2-1\)
\end{itemize}

Con estas operaciones el conjunto \(\mathbb{K}[x]\) presenta una serie
de propiedades importantes que no permiten decir que es un cuerpo. La
condición de cuerpo que nos falla aquí es únicamente que no existe
elemento inverso para todo elemento de \(\mathbb{K}[x]\). De hecho, los
únicos elementos que tienen inverso son los polinomios constantes y
diferentes de 0.

\end{frame}

\begin{frame}{Polinomios}
\protect\hypertarget{polinomios-5}{}

División de polinomios. Dado un polinomio \(p(x)\in\mathbb{K}[x]\), la
división de \(p(x)\) entre otro polinomio \(s(x)\in\mathbb{K}[x]\) de
grado menor o igual al de \(p(x)\), consiste en hallar 2 polinomios
\(q(x),r(x)\in\mathbb{K}[x]\) tales que \[p(x) = s(x)q(x)+r(x)\]

donde el grado de \(r(x)\) es siempre menor que el del divisor \(s(x)\)

\end{frame}

\begin{frame}{Polinomios}
\protect\hypertarget{polinomios-6}{}

Polinomio irreducible. Si es un polinomio con coeficientes en un cuerpo,
no es constante y no se puede descomponer en producto de otros
polinomios sin que esta descomposición sea trivial.

\textbf{Ejemplo 6}

\(p(x) = 1+x^2\in\mathbb{R}[x]\) es irreducible ya que no puede
escribirse de forma \(r(x)s(x)\) con \(r(x),s(x)\in\mathbb{R}[x]\),
\(r(x),s(x)\ne p(x)\) y \(r(x),s(x)\ne 1\).

En cambio, \(q(x)=1-x^2\) no es irreducible, ya que \(q(x)=(1-x)(1+x)\)

\end{frame}

\begin{frame}{Polinomios}
\protect\hypertarget{polinomios-7}{}

Dado un polinomio \(p(x)\in\mathbb{K}[x]\), podemos asociar a \(p(x)\)
una aplicación o función \(\mathbb{K}\longrightarrow\mathbb{K}\)
definida de la manera siguiente: a cada elemento \(\alpha\in\mathbb{K}\)
le hacemos corresponder \(p(\alpha)=a_0+a_1\alpha+\cdots+a_n\alpha^n\).
Esta función es conocida como función evaluadora y el proceso de
sustituir la variable \(x\) por cualquier elemento \(\alpha\) del cuerpo
se lo conoce como evaluar un polinomio en \(\alpha\).

En el caso en que \(\mathbb{K}\) sea un cuerpo infinito, podemos
identificar polinomio con función asociada.

\end{frame}

\begin{frame}{Polinomios}
\protect\hypertarget{polinomios-8}{}

Raíz de un polinomio. Sean \(p(x)\in\mathbb{K}[x]\) y
\(\alpha\in\mathbb{K}\), diremos que \(\alpha\) es raíz de \(p(x)\) si
\(p(\alpha)=0\).

Proposición. \(\alpha\in\mathbb{K}\) es raíz de \(p(x)\)
\(\Leftrightarrow\) \(p(x)=(x-\alpha)q(x)\) con
\(q(x)\in\mathbb{K}[x]\).

De aquí deducimos que si un polinomio de \(\mathbb{K}[x]\) de grado
mayor que 1 tiene una raíz (en \(\mathbb{K}\)) entonces no es
irreducible. El recíproco no es cierto.

\end{frame}

\begin{frame}{Polinomios}
\protect\hypertarget{polinomios-9}{}

\textbf{Demostración}

Primero probaremos la implicación hacia la izquierda,
\(p(x)=(x-\alpha)q(x)\) con
\(q(x)\in\mathbb{K}[x]\Rightarrow\)\(\alpha\in\mathbb{K}\) es raíz de
\(p(x)\)

Evaluando \(p(x)\) en \(\alpha\) tenemos que
\[p(\alpha)=(\alpha-\alpha)q(\alpha) = 0\] lo que, por definición,
implica que \(\alpha\) es raíz de \(p(x)\)

Ahora nos queda demostrar la implicación a la derecha
\(\alpha\in\mathbb{K}\) es raíz de \(p(x)\) \(\Rightarrow\)
\(p(x)=(x-\alpha)q(x)\) con \(q(x)\in\mathbb{K}[x]\)

Si dividimos \(p(x)\) entre \(x-\alpha\), obtenemos
\[p(x) = (x-\alpha)q(x)+r(x)\]

Ahora, por hipótesis tenemos que
\[p(\alpha) = (\alpha-\alpha)q(\alpha)+r(\alpha) = 0 +r(\alpha) = 0\]

\end{frame}

\begin{frame}{Polinomios}
\protect\hypertarget{polinomios-10}{}

Con lo cual \(r(\alpha) = 0\)

Ahora bien, el grado de \(r(x)\) debe ser estrictamente menor al del
divisor, \(x-\alpha\), que es 1. Por tanto, \(r(x)\) es un polinomio
constante. Además, como \(r(\alpha) =0\), tenemos que \[r(x)\equiv 0\]

Así pues, acabamos de demostrar que \(p(x) = (x-\alpha)q(x)\)

\end{frame}

\hypertarget{polinomios-con-r}{%
\section{\texorpdfstring{Polinomios con
\texttt{R}}{Polinomios con R}}\label{polinomios-con-r}}

\begin{frame}[fragile]{Polinomios con \texttt{R}}
\protect\hypertarget{polinomios-con-r-1}{}

Necesitaremos instalar y cargar los paquetes \texttt{polynom} y
\texttt{pracma} para poder utilizar las siguientes funciones.

Para definir un polinomio en \texttt{R}, lo haremos mediante la función
\texttt{polynomial(coef=...)} e igualaremos el parámetro \texttt{coef}
al vector de coeficientes en orden ascendente.

\begin{Shaded}
\begin{Highlighting}[]
\NormalTok{p =}\StringTok{ }\KeywordTok{polynomial}\NormalTok{(}\DataTypeTok{coef =} \KeywordTok{c}\NormalTok{(}\DecValTok{1}\NormalTok{,}\DecValTok{2}\NormalTok{,}\DecValTok{3}\NormalTok{,}\DecValTok{4}\NormalTok{,}\DecValTok{5}\NormalTok{))}
\NormalTok{p}
\end{Highlighting}
\end{Shaded}

\begin{verbatim}
1 + 2*x + 3*x^2 + 4*x^3 + 5*x^4 
\end{verbatim}

\begin{Shaded}
\begin{Highlighting}[]
\NormalTok{q =}\StringTok{ }\KeywordTok{polynomial}\NormalTok{(}\DataTypeTok{coef =} \KeywordTok{c}\NormalTok{(}\DecValTok{1}\NormalTok{,}\DecValTok{2}\NormalTok{,}\DecValTok{1}\NormalTok{))}
\NormalTok{q}
\end{Highlighting}
\end{Shaded}

\begin{verbatim}
1 + 2*x + x^2 
\end{verbatim}

\end{frame}

\begin{frame}[fragile]{Polinomios con \texttt{R}}
\protect\hypertarget{polinomios-con-r-2}{}

Para comprobar si dos polinomios son iguales, utilizamos el operador
lógico \texttt{==}

\begin{Shaded}
\begin{Highlighting}[]
\NormalTok{p }\OperatorTok{==}\StringTok{ }\NormalTok{q}
\end{Highlighting}
\end{Shaded}

\begin{verbatim}
[1] FALSE
\end{verbatim}

Claramente son diferentes, porque, tal y como los hemos definido
anteriormente, ni siquiera tienen el mismo grado

\end{frame}

\begin{frame}[fragile]{Polinomios con \texttt{R}}
\protect\hypertarget{polinomios-con-r-3}{}

Una forma de calcular el grado de un polinomio en \texttt{R} es mediante
la función \texttt{length()} aplicada al polinomio. Eso sí, teniendo en
cuenta que, tal y como hemos definido los polinomios,
\(p(x)=a_0+\cdots+a_nx^n\), estos empiezan en 0. Con lo cual, para
obtener el grado exacto del polinomio, habrá que restar una unidad al
resultado que nos devuelva \texttt{length()}:

\begin{Shaded}
\begin{Highlighting}[]
\NormalTok{gradoP =}\StringTok{ }\KeywordTok{length}\NormalTok{(p)}\OperatorTok{-}\DecValTok{1}
\NormalTok{gradoQ =}\StringTok{ }\KeywordTok{length}\NormalTok{(q)}\OperatorTok{-}\DecValTok{1}
\NormalTok{gradoP}
\end{Highlighting}
\end{Shaded}

\begin{verbatim}
[1] 4
\end{verbatim}

\begin{Shaded}
\begin{Highlighting}[]
\NormalTok{gradoQ}
\end{Highlighting}
\end{Shaded}

\begin{verbatim}
[1] 2
\end{verbatim}

\end{frame}

\begin{frame}[fragile]{Polinomios con \texttt{R}}
\protect\hypertarget{polinomios-con-r-4}{}

Las operaciones suma y producto de polinomios, se llevan a cabo del
siguiente modo:

\begin{Shaded}
\begin{Highlighting}[]
\CommentTok{#Suma}
\NormalTok{p}\OperatorTok{+}\NormalTok{q}
\end{Highlighting}
\end{Shaded}

\begin{verbatim}
2 + 4*x + 4*x^2 + 4*x^3 + 5*x^4 
\end{verbatim}

\begin{Shaded}
\begin{Highlighting}[]
\CommentTok{#Producto de polinomios}
\NormalTok{p}\OperatorTok{*}\NormalTok{q}
\end{Highlighting}
\end{Shaded}

\begin{verbatim}
1 + 4*x + 8*x^2 + 12*x^3 + 16*x^4 + 14*x^5 + 5*x^6 
\end{verbatim}

\end{frame}

\begin{frame}[fragile]{Polinomios con \texttt{R}}
\protect\hypertarget{polinomios-con-r-5}{}

La divisióin de polinomios se realiza mediante \texttt{/}, pero con ello
solo obtenemos el cociente. Para obtener el resto hay que utlizar
\texttt{\%\%}:

\begin{Shaded}
\begin{Highlighting}[]
\NormalTok{cociente =}\StringTok{ }\NormalTok{p }\OperatorTok{/}\StringTok{ }\NormalTok{q}
\NormalTok{resto =}\StringTok{ }\NormalTok{p}\OperatorTok\NormalTok{q}
\NormalTok{cociente}
\end{Highlighting}
\end{Shaded}

\begin{verbatim}
10 - 6*x + 5*x^2 
\end{verbatim}

\begin{Shaded}
\begin{Highlighting}[]
\NormalTok{resto}
\end{Highlighting}
\end{Shaded}

\begin{verbatim}
-9 - 12*x 
\end{verbatim}

\begin{Shaded}
\begin{Highlighting}[]
\NormalTok{q}\OperatorTok{*}\NormalTok{cociente }\OperatorTok{+}\StringTok{ }\NormalTok{resto }\OperatorTok{==}\StringTok{ }\NormalTok{p}
\end{Highlighting}
\end{Shaded}

\begin{verbatim}
[1] TRUE
\end{verbatim}

\end{frame}

\begin{frame}[fragile]{Polinomios con \texttt{R}}
\protect\hypertarget{polinomios-con-r-6}{}

Para evaluar polinomios, utilizaremos la función
\texttt{predict(polinomio,x0)}

\begin{Shaded}
\begin{Highlighting}[]
\KeywordTok{predict}\NormalTok{(p,}\DecValTok{1}\NormalTok{)}
\end{Highlighting}
\end{Shaded}

\begin{verbatim}
[1] 15
\end{verbatim}

\begin{Shaded}
\begin{Highlighting}[]
\KeywordTok{predict}\NormalTok{(q,}\DecValTok{0}\NormalTok{)}
\end{Highlighting}
\end{Shaded}

\begin{verbatim}
[1] 1
\end{verbatim}

\end{frame}

\begin{frame}[fragile]{Polinomios con \texttt{R}}
\protect\hypertarget{polinomios-con-r-7}{}

Para hallar las raíces de un polinomio, podemos utilizar la función
\texttt{polyroot} introduciendo por parámetro el vector de coeficientes
en orden creciente.

\begin{Shaded}
\begin{Highlighting}[]
\CommentTok{#Las raíces del polinomio x^2+2x+1}
\KeywordTok{polyroot}\NormalTok{(}\KeywordTok{c}\NormalTok{(}\DecValTok{1}\NormalTok{,}\DecValTok{2}\NormalTok{,}\DecValTok{1}\NormalTok{))}
\end{Highlighting}
\end{Shaded}

\begin{verbatim}
[1] -1-0i -1+0i
\end{verbatim}

\begin{Shaded}
\begin{Highlighting}[]
\CommentTok{#Las raíces del polinomio x^2-4}
\KeywordTok{polyroot}\NormalTok{(}\KeywordTok{c}\NormalTok{(}\OperatorTok{-}\DecValTok{4}\NormalTok{,}\DecValTok{0}\NormalTok{,}\DecValTok{1}\NormalTok{))}
\end{Highlighting}
\end{Shaded}

\begin{verbatim}
[1]  2+0i -2+0i
\end{verbatim}

Fijaos que \texttt{R} nos devuelve un vector de números complejos a
pesar de que las soluciones en ambos casos son dos números reales

\end{frame}

\hypertarget{polinomios-con-python}{%
\section{\texorpdfstring{Polinomios con
\texttt{Python}}{Polinomios con Python}}\label{polinomios-con-python}}

\begin{frame}[fragile]{Polinomios con \texttt{Python}}
\protect\hypertarget{polinomios-con-python-1}{}

Para definir un polinomio en \texttt{Python}, lo haremos mediante las
funciones \texttt{sympy.symbols()} para indicar con qué variable
trabajamos y \texttt{sympy.Poly()} introduciendo el polinomio por
parámetro:

\begin{Shaded}
\begin{Highlighting}[]
\ImportTok{import}\NormalTok{ sympy }
\NormalTok{x }\OperatorTok{=}\NormalTok{ sympy.symbols(}\StringTok{'x'}\NormalTok{) }
\NormalTok{p }\OperatorTok{=}\NormalTok{ sympy.Poly(x}\OperatorTok{**}\DecValTok{2}\NormalTok{) }
\NormalTok{p}
\end{Highlighting}
\end{Shaded}

\begin{verbatim}
Poly(x**2, x, domain='ZZ')
\end{verbatim}

\begin{Shaded}
\begin{Highlighting}[]
\NormalTok{q }\OperatorTok{=}\NormalTok{ sympy.Poly(}\DecValTok{1}\OperatorTok{+}\NormalTok{x}\OperatorTok{+}\NormalTok{x}\OperatorTok{**}\DecValTok{3}\NormalTok{) }
\NormalTok{q}
\end{Highlighting}
\end{Shaded}

\begin{verbatim}
Poly(x**3 + x + 1, x, domain='ZZ')
\end{verbatim}

\end{frame}

\begin{frame}[fragile]{Polinomios con \texttt{Python}}
\protect\hypertarget{polinomios-con-python-2}{}

O bien, otra forma de definir polinomios es mediante la librería
\texttt{numpy}, introduciendo como parámetro el vector de coeficientes
en orden descendente

\begin{Shaded}
\begin{Highlighting}[]
\ImportTok{import}\NormalTok{ numpy}
\NormalTok{r }\OperatorTok{=}\NormalTok{ numpy.poly1d([}\DecValTok{1}\NormalTok{,}\DecValTok{2}\NormalTok{,}\DecValTok{1}\NormalTok{]) }
\BuiltInTok{print}\NormalTok{(r)}
\end{Highlighting}
\end{Shaded}

\begin{verbatim}
   2
1 x + 2 x + 1
\end{verbatim}

\begin{Shaded}
\begin{Highlighting}[]
\NormalTok{s }\OperatorTok{=}\NormalTok{ numpy.poly1d([}\DecValTok{1}\NormalTok{,}\DecValTok{2}\NormalTok{,}\DecValTok{3}\NormalTok{,}\DecValTok{4}\NormalTok{,}\DecValTok{5}\NormalTok{])}
\BuiltInTok{print}\NormalTok{(s)}
\end{Highlighting}
\end{Shaded}

\begin{verbatim}
   4     3     2
1 x + 2 x + 3 x + 4 x + 5
\end{verbatim}

\end{frame}

\begin{frame}[fragile]{Polinomios con \texttt{Python}}
\protect\hypertarget{polinomios-con-python-3}{}

Para comprobar si dos polinomios son iguales, utilizamos el operador
lógico \texttt{==}

\begin{Shaded}
\begin{Highlighting}[]
\NormalTok{p }\OperatorTok{==}\NormalTok{ q  }
\end{Highlighting}
\end{Shaded}

\begin{verbatim}
False
\end{verbatim}

\begin{Shaded}
\begin{Highlighting}[]
\NormalTok{r }\OperatorTok{==}\NormalTok{ s}
\end{Highlighting}
\end{Shaded}

\begin{verbatim}
False
\end{verbatim}

Claramente son diferentes, porque, tal y como los hemos definido
anteriormente, ni siquiera tienen el mismo grado.

\end{frame}

\begin{frame}{Polinomios con \texttt{Python}}
\protect\hypertarget{polinomios-con-python-4}{}

Observación. Fijaos que no podemos comparar los polinomios creados con
librerías diferentes, ya que da error al ser diferentes tipos de objeto.

\end{frame}

\begin{frame}[fragile]{Polinomios con \texttt{Python}}
\protect\hypertarget{polinomios-con-python-5}{}

Para calcular el grado de cualquier polinomio en \texttt{Python}, lo
haremos utilizando \texttt{Polynomial.degree()}

\begin{Shaded}
\begin{Highlighting}[]
\NormalTok{p.degree()}
\end{Highlighting}
\end{Shaded}

\begin{verbatim}
2
\end{verbatim}

\begin{Shaded}
\begin{Highlighting}[]
\NormalTok{q.degree()}
\end{Highlighting}
\end{Shaded}

\begin{verbatim}
3
\end{verbatim}

\end{frame}

\begin{frame}[fragile]{Polinomios con \texttt{Python}}
\protect\hypertarget{polinomios-con-python-6}{}

o, si estamos trabajando con la librería \texttt{numpy}, lo hacemos
mediante la función \texttt{Polynomial.order}

\begin{Shaded}
\begin{Highlighting}[]
\NormalTok{r }\OperatorTok{=}\NormalTok{ numpy.poly1d([}\DecValTok{1}\NormalTok{,}\DecValTok{2}\NormalTok{,}\DecValTok{1}\NormalTok{]) }
\NormalTok{s }\OperatorTok{=}\NormalTok{ numpy.poly1d([}\DecValTok{1}\NormalTok{,}\DecValTok{2}\NormalTok{,}\DecValTok{3}\NormalTok{,}\DecValTok{4}\NormalTok{,}\DecValTok{5}\NormalTok{])}
\NormalTok{r.order}
\end{Highlighting}
\end{Shaded}

\begin{verbatim}
2
\end{verbatim}

\begin{Shaded}
\begin{Highlighting}[]
\NormalTok{s.order}
\end{Highlighting}
\end{Shaded}

\begin{verbatim}
4
\end{verbatim}

¡Ojo! Fijaos que hemos tenido que volver a definir los polinomios. De no
estar, la consola nos devolvería error.

\end{frame}

\begin{frame}[fragile]{Polinomios con \texttt{Python}}
\protect\hypertarget{polinomios-con-python-7}{}

Las operaciones suma y producto de polinomios, se llevan a cabo del
siguiente modo:

\begin{Shaded}
\begin{Highlighting}[]
\NormalTok{p}\OperatorTok{+}\NormalTok{q }
\end{Highlighting}
\end{Shaded}

\begin{verbatim}
Poly(x**3 + x**2 + x + 1, x, domain='ZZ')
\end{verbatim}

\begin{Shaded}
\begin{Highlighting}[]
\NormalTok{p}\OperatorTok{*}\NormalTok{q }
\end{Highlighting}
\end{Shaded}

\begin{verbatim}
Poly(x**5 + x**3 + x**2, x, domain='ZZ')
\end{verbatim}

\end{frame}

\begin{frame}[fragile]{Polinomios con \texttt{Python}}
\protect\hypertarget{polinomios-con-python-8}{}

y, con la librería \texttt{numpy}, la suma y el producto de polinomios
se realizan del siguiente modo:

\begin{Shaded}
\begin{Highlighting}[]
\NormalTok{r }\OperatorTok{=}\NormalTok{ numpy.poly1d([}\DecValTok{1}\NormalTok{,}\DecValTok{2}\NormalTok{,}\DecValTok{1}\NormalTok{]) }
\NormalTok{s }\OperatorTok{=}\NormalTok{ numpy.poly1d([}\DecValTok{1}\NormalTok{,}\DecValTok{2}\NormalTok{,}\DecValTok{3}\NormalTok{,}\DecValTok{4}\NormalTok{,}\DecValTok{5}\NormalTok{])}
\NormalTok{r}\OperatorTok{+}\NormalTok{s}
\end{Highlighting}
\end{Shaded}

\begin{verbatim}
poly1d([1, 2, 4, 6, 6])
\end{verbatim}

\begin{Shaded}
\begin{Highlighting}[]
\NormalTok{r}\OperatorTok{*}\NormalTok{s}
\end{Highlighting}
\end{Shaded}

\begin{verbatim}
poly1d([ 1,  4,  8, 12, 16, 14,  5])
\end{verbatim}

\end{frame}

\begin{frame}[fragile]{Polinomios con \texttt{Python}}
\protect\hypertarget{polinomios-con-python-9}{}

La división de polinomios la obtenemos tal y como se muestra a
continuación (solo utilizando la librería \texttt{numpy}):

\begin{Shaded}
\begin{Highlighting}[]
\NormalTok{r }\OperatorTok{=}\NormalTok{ numpy.poly1d([}\DecValTok{1}\NormalTok{,}\DecValTok{2}\NormalTok{,}\DecValTok{1}\NormalTok{]) }
\NormalTok{s }\OperatorTok{=}\NormalTok{ numpy.poly1d([}\DecValTok{1}\NormalTok{,}\DecValTok{2}\NormalTok{,}\DecValTok{3}\NormalTok{,}\DecValTok{4}\NormalTok{,}\DecValTok{5}\NormalTok{])}
\NormalTok{s}\OperatorTok{/}\NormalTok{r}
\end{Highlighting}
\end{Shaded}

\begin{verbatim}
(poly1d([1., 0., 2.]), poly1d([3.]))
\end{verbatim}

\begin{Shaded}
\begin{Highlighting}[]
\NormalTok{r}\OperatorTok{*}\NormalTok{numpy.poly1d([}\DecValTok{1}\NormalTok{,}\DecValTok{0}\NormalTok{,}\DecValTok{2}\NormalTok{])}\OperatorTok{+}\DecValTok{3} \OperatorTok{==}\NormalTok{ s}
\end{Highlighting}
\end{Shaded}

\begin{verbatim}
True
\end{verbatim}

Observad que primero se devuelve el cociente y, a continuación, el resto
de la división

\end{frame}

\begin{frame}[fragile]{Polinomios con \texttt{Python}}
\protect\hypertarget{polinomios-con-python-10}{}

Para evaluar polinomios, haciendo uso de la librería \texttt{numpy}, lo
hacemos del siguiente modo:

\begin{Shaded}
\begin{Highlighting}[]
\NormalTok{r }\OperatorTok{=}\NormalTok{ numpy.poly1d([}\DecValTok{1}\NormalTok{,}\DecValTok{2}\NormalTok{,}\DecValTok{1}\NormalTok{]) }
\NormalTok{s }\OperatorTok{=}\NormalTok{ numpy.poly1d([}\DecValTok{1}\NormalTok{,}\DecValTok{2}\NormalTok{,}\DecValTok{3}\NormalTok{,}\DecValTok{4}\NormalTok{,}\DecValTok{5}\NormalTok{])}
\NormalTok{r(}\DecValTok{0}\NormalTok{)}
\end{Highlighting}
\end{Shaded}

\begin{verbatim}
1
\end{verbatim}

\begin{Shaded}
\begin{Highlighting}[]
\NormalTok{s(}\DecValTok{2}\NormalTok{)}
\end{Highlighting}
\end{Shaded}

\begin{verbatim}
57
\end{verbatim}

\end{frame}

\begin{frame}[fragile]{Polinomios con \texttt{Python}}
\protect\hypertarget{polinomios-con-python-11}{}

Para encontrar las raíces de polinomios, haciendo uso de la librería
\texttt{numpy}, lo hacemos utilizando \texttt{Polynomial.r}:

\begin{Shaded}
\begin{Highlighting}[]
\NormalTok{r }\OperatorTok{=}\NormalTok{ numpy.poly1d([}\DecValTok{1}\NormalTok{,}\DecValTok{2}\NormalTok{,}\DecValTok{1}\NormalTok{]) }
\NormalTok{s }\OperatorTok{=}\NormalTok{ numpy.poly1d([}\DecValTok{1}\NormalTok{,}\DecValTok{2}\NormalTok{,}\DecValTok{3}\NormalTok{,}\DecValTok{4}\NormalTok{,}\DecValTok{5}\NormalTok{])}
\NormalTok{r.r}
\end{Highlighting}
\end{Shaded}

\begin{verbatim}
array([-1., -1.])
\end{verbatim}

\begin{Shaded}
\begin{Highlighting}[]
\NormalTok{s.r}
\end{Highlighting}
\end{Shaded}

\begin{verbatim}
array([-1.28781548+0.85789676j, -1.28781548-0.85789676j,
        0.28781548+1.41609308j,  0.28781548-1.41609308j])
\end{verbatim}

\end{frame}

\hypertarget{polinomios-con-octave}{%
\section{\texorpdfstring{Polinomios con
\texttt{Octave}}{Polinomios con Octave}}\label{polinomios-con-octave}}

\begin{frame}[fragile]{Polinomios con \texttt{Octave}}
\protect\hypertarget{polinomios-con-octave-1}{}

Para definir un polinomio en \texttt{Octave}, lo haremos mediante
vectores. Para mostrarlos simbólicamente, utilizaremos la función
\texttt{polyout(vector,\textquotesingle{}variable\textquotesingle{})},
donde las entradas del vector son los coeficientes en orden descendente.

\begin{Shaded}
\begin{Highlighting}[]
\NormalTok{p }\OperatorTok{=}\NormalTok{ [}\FloatTok{1}\OperatorTok{,}\FloatTok{2}\OperatorTok{,}\FloatTok{3}\OperatorTok{,}\FloatTok{4}\OperatorTok{,}\FloatTok{5}\NormalTok{]}\OperatorTok{;} 
\FunctionTok{polyout}\NormalTok{(p}\OperatorTok{,}\StringTok{'x'}\NormalTok{) }
\NormalTok{q }\OperatorTok{=}\NormalTok{ [}\FloatTok{1}\OperatorTok{,}\FloatTok{0}\OperatorTok{,}\FloatTok{0}\OperatorTok{,}\FloatTok{1}\NormalTok{]}\OperatorTok{;}
\FunctionTok{polyout}\NormalTok{(q}\OperatorTok{,}\StringTok{'x'}\NormalTok{)}
\end{Highlighting}
\end{Shaded}

\begin{verbatim}
1*x^4 + 2*x^3 + 3*x^2 + 4*x^1 + 5
1*x^3 + 0*x^2 + 0*x^1 + 1
\end{verbatim}

\end{frame}

\begin{frame}[fragile]{Polinomios con \texttt{Octave}}
\protect\hypertarget{polinomios-con-octave-2}{}

Una forma de calcular el grado de un polinomio en \texttt{Octave} es
mediante la función \texttt{length()} aplicada al vector de
coeficientes. Eso sí, teniendo en cuenta que, tal y como hemos definido
los polinomios, \(p(x)=a_0+\cdots+a_nx^n\), estos empiezan en 0. Con lo
cual, para obtener el grado exacto del polinomio, habrá que restar una
unidad al resultado que nos devuelva \texttt{length()}:

\begin{Shaded}
\begin{Highlighting}[]
\NormalTok{p }\OperatorTok{=}\NormalTok{ [}\FloatTok{1}\OperatorTok{,}\FloatTok{2}\OperatorTok{,}\FloatTok{3}\OperatorTok{,}\FloatTok{4}\OperatorTok{,}\FloatTok{5}\NormalTok{]}\OperatorTok{;} 
\NormalTok{q }\OperatorTok{=}\NormalTok{ [}\FloatTok{1}\OperatorTok{,}\FloatTok{0}\OperatorTok{,}\FloatTok{0}\OperatorTok{,}\FloatTok{1}\NormalTok{]}\OperatorTok{;}
\NormalTok{gradoP }\OperatorTok{=} \FunctionTok{length}\NormalTok{(p)}\OperatorTok{-}\FloatTok{1}
\NormalTok{gradoQ }\OperatorTok{=} \FunctionTok{length}\NormalTok{(q)}\OperatorTok{-}\FloatTok{1}
\end{Highlighting}
\end{Shaded}

\begin{verbatim}
gradoP =  4
gradoQ =  3
\end{verbatim}

\end{frame}

\begin{frame}[fragile]{Polinomios con \texttt{Octave}}
\protect\hypertarget{polinomios-con-octave-3}{}

Como, en \texttt{Octave}, los polinomios vienen representados por
vectores, sumar polinomios no es una operación directa en
\texttt{Octave}, ya que la mayoría de veces nos encontraremos con
vectores de diferente longitud. En esos casos, se nos devolverá error
por consola.

Lo mismo ocurre si quisiésemos comprobar que dos polinomios de diferente
grado son iguales

\end{frame}

\begin{frame}[fragile]{Polinomios con \texttt{Octave}}
\protect\hypertarget{polinomios-con-octave-4}{}

Una solución a este problema es rellenar con 0 el vector de coeficientes
hasta alcanzar la máxima longitud de los vectores que tengamos. Esto lo
podemos conseguir mediante la función \texttt{zeros()}:

\begin{Shaded}
\begin{Highlighting}[]
\NormalTok{p }\OperatorTok{=}\NormalTok{ [}\FloatTok{1}\OperatorTok{,}\FloatTok{2}\OperatorTok{,}\FloatTok{3}\OperatorTok{,}\FloatTok{4}\OperatorTok{,}\FloatTok{5}\NormalTok{]}\OperatorTok{;}\NormalTok{ q }\OperatorTok{=}\NormalTok{ [}\FloatTok{1}\OperatorTok{,}\FloatTok{0}\OperatorTok{,}\FloatTok{0}\OperatorTok{,}\FloatTok{1}\NormalTok{]}\OperatorTok{;}\NormalTok{ gradoP }\OperatorTok{=} \FunctionTok{length}\NormalTok{(p)}\OperatorTok{-}\FloatTok{1}\OperatorTok{;}\NormalTok{ gradoQ }\OperatorTok{=} \FunctionTok{length}\NormalTok{(q)}\OperatorTok{-}\FloatTok{1}\OperatorTok{;}
\NormalTok{p }\OperatorTok{=}\NormalTok{ [}\FunctionTok{zeros}\NormalTok{(}\FloatTok{1}\OperatorTok{,}\NormalTok{ gradoQ}\OperatorTok{-}\NormalTok{gradoP)}\OperatorTok{,}\NormalTok{ p]}\OperatorTok{,}\NormalTok{ q }\OperatorTok{=}\NormalTok{ [}\FunctionTok{zeros}\NormalTok{(}\FloatTok{1}\OperatorTok{,}\NormalTok{ gradoP}\OperatorTok{-}\NormalTok{gradoQ)}\OperatorTok{,}\NormalTok{ q]}\OperatorTok{,}\NormalTok{ suma }\OperatorTok{=}\NormalTok{ p}\OperatorTok{+}\NormalTok{q}
\end{Highlighting}
\end{Shaded}

\begin{verbatim}
p =

   1   2   3   4   5

q =

   0   1   0   0   1

suma =

   1   3   3   4   6
\end{verbatim}

\end{frame}

\begin{frame}[fragile]{Polinomios con \texttt{Octave}}
\protect\hypertarget{polinomios-con-octave-5}{}

Lo que sí podemos hacer es multiplicar polinomios mediante la función
\texttt{conv}

\begin{Shaded}
\begin{Highlighting}[]
\NormalTok{p }\OperatorTok{=}\NormalTok{ [}\FloatTok{1}\OperatorTok{,}\FloatTok{2}\OperatorTok{,}\FloatTok{3}\OperatorTok{,}\FloatTok{4}\OperatorTok{,}\FloatTok{5}\NormalTok{]}\OperatorTok{;}
\NormalTok{q }\OperatorTok{=}\NormalTok{ [}\FloatTok{1}\OperatorTok{,}\FloatTok{0}\OperatorTok{,}\FloatTok{0}\OperatorTok{,}\FloatTok{1}\NormalTok{]}\OperatorTok{;}
\NormalTok{producto }\OperatorTok{=} \FunctionTok{conv}\NormalTok{(p}\OperatorTok{,}\NormalTok{q)}\OperatorTok{;}
\FunctionTok{polyout}\NormalTok{(producto}\OperatorTok{,}\StringTok{'x'}\NormalTok{)}
\end{Highlighting}
\end{Shaded}

\begin{verbatim}
1*x^7 + 2*x^6 + 3*x^5 + 5*x^4 + 7*x^3 + 3*x^2 + 4*x^1 + 5
\end{verbatim}

\end{frame}

\begin{frame}[fragile]{Polinomios con \texttt{Octave}}
\protect\hypertarget{polinomios-con-octave-6}{}

La división de polinomios se consigue mediante la función
\texttt{deconv(numerador,denominador)}. Esta función devuelve los
vecotres de coeficientes del cociente y el resto de la división
polinómica, en orden descendente:

\begin{Shaded}
\begin{Highlighting}[]
\NormalTok{p }\OperatorTok{=}\NormalTok{ [}\FloatTok{1}\OperatorTok{,}\FloatTok{2}\OperatorTok{,}\FloatTok{3}\OperatorTok{,}\FloatTok{4}\OperatorTok{,}\FloatTok{5}\NormalTok{]}\OperatorTok{;}
\NormalTok{q }\OperatorTok{=}\NormalTok{ [}\FloatTok{1}\OperatorTok{,}\FloatTok{0}\OperatorTok{,}\FloatTok{0}\OperatorTok{,}\FloatTok{1}\NormalTok{]}\OperatorTok{;}
\NormalTok{[cociente}\OperatorTok{,}\NormalTok{ resto] }\OperatorTok{=} \FunctionTok{deconv}\NormalTok{(p}\OperatorTok{,}\NormalTok{q)}
\end{Highlighting}
\end{Shaded}

\begin{verbatim}
cociente =

   1   2

resto =

   0   0   3   3   3
\end{verbatim}

\end{frame}

\begin{frame}[fragile]{Polinomios con \texttt{Octave}}
\protect\hypertarget{polinomios-con-octave-7}{}

Fijaos que el vector de coeficientes del resto tiene la misma longitud
que el dividendo, con lo cual podemos realizar la comprobación de la
división:

\begin{Shaded}
\begin{Highlighting}[]
\NormalTok{p }\OperatorTok{=}\NormalTok{ [}\FloatTok{1}\OperatorTok{,}\FloatTok{2}\OperatorTok{,}\FloatTok{3}\OperatorTok{,}\FloatTok{4}\OperatorTok{,}\FloatTok{5}\NormalTok{]}\OperatorTok{;}
\NormalTok{q }\OperatorTok{=}\NormalTok{ [}\FloatTok{1}\OperatorTok{,}\FloatTok{0}\OperatorTok{,}\FloatTok{0}\OperatorTok{,}\FloatTok{1}\NormalTok{]}\OperatorTok{;}
\NormalTok{[cociente}\OperatorTok{,}\NormalTok{ resto] }\OperatorTok{=} \FunctionTok{deconv}\NormalTok{(p}\OperatorTok{,}\NormalTok{q)}\OperatorTok{;}
\FunctionTok{length}\NormalTok{(resto) }\OperatorTok{==} \FunctionTok{length}\NormalTok{(p)}
\NormalTok{p }\OperatorTok{==} \FunctionTok{conv}\NormalTok{(q}\OperatorTok{,}\NormalTok{cociente) }\OperatorTok{+}\NormalTok{ resto}
\end{Highlighting}
\end{Shaded}

\begin{verbatim}
ans = 1
ans =

  1  1  1  1  1
\end{verbatim}

donde los unos representan el valor lógico \texttt{True}

\end{frame}

\begin{frame}[fragile]{Polinomios con \texttt{Octave}}
\protect\hypertarget{polinomios-con-octave-8}{}

Para evaluar polinomios, utilizamos la función
\texttt{polyval(polinomio,x0)}:

\begin{Shaded}
\begin{Highlighting}[]
\NormalTok{p }\OperatorTok{=}\NormalTok{ [}\FloatTok{1}\OperatorTok{,}\FloatTok{2}\OperatorTok{,}\FloatTok{3}\OperatorTok{,}\FloatTok{4}\OperatorTok{,}\FloatTok{5}\NormalTok{]}\OperatorTok{;}
\NormalTok{q }\OperatorTok{=}\NormalTok{ [}\FloatTok{1}\OperatorTok{,}\FloatTok{0}\OperatorTok{,}\FloatTok{0}\OperatorTok{,}\FloatTok{1}\NormalTok{]}\OperatorTok{;}
\FunctionTok{polyval}\NormalTok{(p}\OperatorTok{,}\FloatTok{0}\NormalTok{) }
\FunctionTok{polyval}\NormalTok{(q}\OperatorTok{,}\FloatTok{3}\NormalTok{)}
\end{Highlighting}
\end{Shaded}

\begin{verbatim}
ans =  5
ans =  28
\end{verbatim}

\end{frame}

\begin{frame}[fragile]{Polinomios con \texttt{Octave}}
\protect\hypertarget{polinomios-con-octave-9}{}

Para hallar las raíces de un polinomio, hacemos uso de la función
\texttt{roots()}:

\begin{Shaded}
\begin{Highlighting}[]
\NormalTok{p }\OperatorTok{=}\NormalTok{ [}\FloatTok{1}\OperatorTok{,}\FloatTok{2}\OperatorTok{,}\FloatTok{3}\OperatorTok{,}\FloatTok{4}\OperatorTok{,}\FloatTok{5}\NormalTok{]}\OperatorTok{;}
\NormalTok{q }\OperatorTok{=}\NormalTok{ [}\FloatTok{1}\OperatorTok{,}\FloatTok{0}\OperatorTok{,}\FloatTok{1}\NormalTok{]}\OperatorTok{;}
\FunctionTok{roots}\NormalTok{(p)}
\FunctionTok{roots}\NormalTok{(q)}
\end{Highlighting}
\end{Shaded}

\begin{verbatim}
ans =

  -1.28782 + 0.85790i
  -1.28782 - 0.85790i
   0.28782 + 1.41609i
   0.28782 - 1.41609i

ans =

  -0 + 1i
   0 - 1i
\end{verbatim}

\end{frame}

\hypertarget{el-principio-de-inducciuxf3n}{%
\section{El principio de inducción}\label{el-principio-de-inducciuxf3n}}

\begin{frame}{El principio de inducción}
\protect\hypertarget{el-principio-de-inducciuxf3n-1}{}

Principio de Inducción. El principio de inducción afirma que si \(P(n)\)
es una propiedad sobre \(n\in\mathbb{N}\) y se cumple que

\begin{itemize}
\tightlist
\item
  \(P(1)\) es cierta (Caso base)
\item
  Si \(P(n)\) es cierta, entonces \(P(n+1)\) es cierta (Caso inductivo)
\end{itemize}

entonces \(P(n)\) es cierta para todo \(n\in\mathbb{N}\)

\end{frame}

\begin{frame}{El principio de inducción}
\protect\hypertarget{el-principio-de-inducciuxf3n-2}{}

El principio de Inducción también es válido si \(n\in\mathbb{Z}\) del
siguiente modo:

Si \(P(n)\) es una propiedad sobre \(n\in\mathbb{Z}\), con
\(n_0\in\mathbb{Z}\) y se cumple que

\begin{itemize}
\tightlist
\item
  \(P(n_0)\) es cierta
\item
  Si \(P(n)\) es cierta, entonces \(P(n+1)\) es cierta
\end{itemize}

entonces \(P(n)\) es cierta para todo \(n\in\mathbb{Z},\ n\ge n_0\)

\end{frame}

\begin{frame}{El principio de inducción}
\protect\hypertarget{el-principio-de-inducciuxf3n-3}{}

Principio de Inducción completa. Si \(P(n)\) es una propiedad sobre
\(n\in\mathbb{N}\) y se cumple que

\begin{itemize}
\tightlist
\item
  \(P(1)\) es cierta (Caso base)
\item
  Si \(P(n)\) es cierta para \(1,2,\dots, n\), entonces \(P(n+1)\) es
  cierta (Caso inductivo)
\end{itemize}

entonces \(P(n)\) es cierta para todo \(n\in\mathbb{N}\)

\end{frame}

\begin{frame}{El principio de inducción}
\protect\hypertarget{el-principio-de-inducciuxf3n-4}{}

El principio de Inducción completa también es válido si
\(n\in\mathbb{Z}\) del siguiente modo:

Si \(P(n)\) es una propiedad sobre \(n\in\mathbb{Z}\),
\(n_0\in\mathbb{Z}\) y se cumple que

\begin{itemize}
\tightlist
\item
  \(P(n_0)\) es cierta
\item
  Si \(P(n)\) es cierta para \(n_0\le n\), entonces \(P(n+1)\) es cierta
\end{itemize}

entonces \(P(n)\) es cierta para todo \(n\in\mathbb{Z}\ n\ge n_0\)

\end{frame}

\end{document}
